% Awesome Source CV LaTeX Template
%
% This template has been downloaded from:
% https://github.com/darwiin/awesome-neue-latex-cv
%
% Author:
% Christophe Roger
%
% Template license:
% CC BY-SA 4.0 (https://creativecommons.org/licenses/by-sa/4.0/)

%Section: Project
\sectionTitle{Projects}{\faLaptop}

\begin{projects}
	\projectsimple
	{Automated simplification and summarization, applied to legal documents}{2018 - Current}
	{Currently writing proposals for model that will simultaneously simplify and summarize a legal document for a non-legal audience as well as the creation of a new legal data set. The model is inspired by a semantic approach to simplification, similar to that of Narayan and Gardent (2014). }

	\projectnolink
	{Writer's aid for South Bolivian Quechua [quh], Kalallusta variety}{2018 - Current}
	{A character level n-gram writer's aid that detects words that do not conform to standard Kalallusta morphology as a term project for a Language Documentation class. This variety of Quechua is severely under-documented and does not currently have any writers aids. }
	{Python36, Pycharm}

	\projectnolink
	{Mention Generation using Variational Autoencoders}{2018 - Current}
	{Working with another student to develop a multi-decoder Variational Autoencoder (VAE) that produces an entity mention conditioned on type. This is for a final project in a Natural Language Processing class.  }
	{Python35, Pytorch, CONLL2012, git, GitHub, Pycharm}


	\project
	{The Verb Lexicon as Revealed by Recursive LDA}{2013 - 2015}
	{\website{http://psychology.emory.edu/cognition/wolff/research.html}{http://psychology.emory.edu/cognition/wolff/research.html} }
	{Worked with another student to develop a program based on a the Stanford Factored English Parser to find patterns in english syntax in relation to verbs. Developed an experiment which sought patterns of similarity between groups of words.}
	{Java, Python27, Eclipse, AMT, HTML/CSS, SQL, SurveyMonkey}
				
	\project
	{A Computational Approach to the Recovery of Primitive Concepts}{2014}
	{ \website{https://qa-etd.library.emory.edu/concern/etds/m900nv052?locale=en}{Emory ETD}}
	{Developed a Python module utilizing JSON to implement a recursive algorithm based on the traditional linguistic theory of Primitive Concepts. Compared results to various prominent primitive concepts sets.}
	{Python27, Liclipse, NLTK, JSON}

\end{projects}
